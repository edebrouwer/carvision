\documentclass{beamer}

% \usepackage{beamerthemesplit} // Activate for custom appearance

\usetheme{Madrid}
\title{CarVision}
\subtitle{Car Damage Visual Detection}
\author{Edward De Brouwer}
\titlegraphic{\includegraphics[width=6cm]{front.png}}

\date{\today}

\AtBeginSection[]
{
  \begin{frame}
  \frametitle{Contents}
  \tableofcontents[currentsection, hideothersubsections, pausesubsections]
  \end{frame} 
}

\begin{document}

\frame{\titlepage}

\section[Outline]{}
\frame{\tableofcontents}

\section{Introduction}

\frame{
\frametitle{Goals and Objectives}

Main objective is to explore the potential of computer vision for car crash assistance. \\~\\

Scenario : after a damage, the insured takes a picture of the car :
\begin{itemize}
\item<1-> Automatic estimation of the costs
\item<2-> Orientation towards the most suited garage or bodyshop
\item<3-> Better handling of the claim. Quick and efficient
\item<4-> Mitigation of the fraud \\~\\
\end{itemize}

\pause
\pause
\pause 
\pause

May be the topic of an hackathon hosted by AXA Belgium. 

}


\section{The Data}

\frame{
\frametitle{Dataset}
Bad news : there is no available dataset ! \\~\\

Possible tracks :
\begin{itemize}
\item<1-> Informex : third party that collects and stores data about all the claims.
	\begin{itemize}
		\item No automatic way of fetching the data 
		\item Confidentiality Issues
		\item May be charged (Third Party)
		\item Format : pictures embedded in pdfs files...
	\end{itemize}
	
\item<2-> DIY : The winding road !
\end{itemize}
}

\frame{
\frametitle{Dataset}

Scrapping of cars pictures over 3 big second hand car online retailers :

\begin{itemize}
\item Autolive.be : Damaged and Clean Cars
\item Eraiparables.com : Damaged and Clean Cars
\item Carexportamerica.com : Mainly Clean Cars \\~\\
\end{itemize}

Cropping (Google Vision API assisted) and resizing to 400x250 px. \\~\\

Manual labelling through a computer assisted framework :
\begin{itemize}
\item Clean Car
\item Damaged Car
\item Remove 
\end{itemize}
}

\frame{
\frametitle{Dataset}
\framesubtitle{Example : Damaged Car}

\begin{figure}[htbp]
   \centering
   \includegraphics[width=12cm]{damaged.png} % requires the graphicx package
\end{figure}
}


\frame{
\frametitle{Dataset}
\framesubtitle{Example : Clean Car}

\begin{figure}[htbp]
   \centering
   \includegraphics[width=12cm]{clean.png} % requires the graphicx package
\end{figure}
}

\frame{
\frametitle{Dataset}
\framesubtitle{Example : Removed Car}

\begin{figure}[htbp]
   \centering
   \includegraphics[width=12cm]{remove.png} % requires the graphicx package
\end{figure}
}

\frame{
\frametitle{Dataset}

1066 retained pictures among which :

\begin{itemize}
\item 444 Clean Cars
\item 622 Damaged Cars \\~\\
\end{itemize}

Separated in 3 folders as :

\begin{itemize}
\item TrainSet : $70\%$
\item ValidationSet: $20\%$
\item TestSet: $10\%$
\end{itemize}

Percentage of Damaged in each set is kept constant : $58 \%$ (baseline) \\~\\

With Data Augmentation techniques, the data set was inflated to 3500 images.
}

\section{Deep Convolutional Neural Networks}

\frame{
\frametitle{ConvNets}
\framesubtitle{Convolutional Layer}

Convolutional Neural Networks: makes use of spatial organization of the data. \\~\\

Main component : the convolutional layer :

\begin{figure}[htbp]
   \centering
   \includegraphics[width=10cm]{conv.png} % requires the graphicx package
\end{figure}
}


\frame{
\frametitle{ConvNets}
\framesubtitle{Stacked Together}

Each layer detects increasingly complex features in the image. \\~\\
Spatial dimension is reduced progressively through the network.

\begin{figure}[htbp]
   \centering
   \includegraphics[width=7cm]{convnet.png} % requires the graphicx package
\end{figure}

A classifier (or regression) is then stacked on top of the last features.
}

\frame{
\frametitle{ConvNets}
\frametitle{Performance}
Image Net Challenge : Correctly classify pictures among 1000 labels. \\~\\

Top-5 error : 
\begin{figure}[htbp]
   \centering
   \includegraphics[width=7cm]{ilsvrc.png} % requires the graphicx package
\end{figure}

Team from Nanjing University won the 2017 Challenge (Detection).
}


\frame{
\frametitle{ConvNets}
\frametitle{Performance}
Combined with other network structures (RNN), we can achieve impressive results :

\begin{figure}[htbp]
   \centering
   \includegraphics[width=10cm]{captioning.png} % requires the graphicx package
\end{figure}
}


\section{Transfer Learning}

\frame{
\frametitle{GoogleNet InceptionV3}

Research teams already successfully trained deep networks for detection. \\~\\

Idea : re-use part of their network and specialize in on apart specific task.

I chose the GoogleNet InceptionV3 network :

\begin{itemize}
\item Winner of the 2014 ILSVRC
\item $5.6\%$ Top-5 error
\item 42 layers deep
\item Weights are public and easily available through Keras
\item Took weeks to train on GPUs
\end{itemize}

\begin{flushright}
And it looks like ...
\end{flushright}
}

\frame{
\frametitle{GoogleNet InceptionV3}
...this :

\begin{figure}[htbp]
   \centering
   \includegraphics[width=10cm]{v3.png} % requires the graphicx package
\end{figure}
}

\frame{
\frametitle{GoogleNet InceptionV3}
\framesubtitle{Scavenging a Deep Network}
Lower layers are supposed to learn basic features of the pictures

\begin{figure}[htbp]
   \centering
   \includegraphics[width=10cm]{transferlearning.png} % requires the graphicx package
\end{figure}

Add customized upper layers (classifiers) upon the top features ! \\~\\
}

\frame{
\frametitle{GoogleNet InceptionV3}
\framesubtitle{Scavenging a Deep Network}

\begin{figure}[htbp]
   \centering
   \includegraphics[width=10cm]{transferlearning.png} % requires the graphicx package
\end{figure}

Results :
\begin{itemize}
\item False Positive : $16\%$  and    False Negative : $22\%$
\item Overal Performance : $81\%$
\end{itemize}

If we use an SVM instead :
\begin{itemize}
\item Overal Performance : $79\%$
\end{itemize}
}

\frame{
\frametitle{GoogleNet InceptionV3}
\framesubtitle{Scavenging a Deep Network}

Let's do better !

\begin{figure}[htbp]
   \centering
   \includegraphics[width=10cm]{transfer2.png} % requires the graphicx package
\end{figure}

Results :
\begin{itemize}
\item False Positive : $1.5\%$  and    False Negative : $10.2\%$
\item Overal Performance : $93.47\%$
\end{itemize}
}

\section{Visualization and Damage Localization}

\frame{
\frametitle{Visualization}
\framesubtitle{Deep Dreams...}

We wish to have better insight on how our model is working. \\~\\

Different parts of the network to visualize :

\begin{itemize}
\item Feature Maps can be visualized as grey-scale pictures.
\item Filters are more difficult as their dimension depends on the depth of the previous layer...
\end{itemize}

Idea : compute the image that maximizes the output of a specific filter through gradient ascent :

\begin{table}
\centering
\begin{tabular}{cc}
   \includegraphics[width=5cm]{filter1.png} % requires the graphicx package
&
   \includegraphics[width=5cm]{filter2.png} % requires the graphicx package
\\
\end{tabular}
\end{table}
}

\frame{
\frametitle{Visualization}
\framesubtitle{Network's attention : Class Activation Map}

What if we can visualize what the network is focusing on ? \\~\\

Idea : retrieve activations of last convolutional layer.


\begin{figure}[htbp]
   \centering
   \includegraphics[width=4cm]{gap.png} % requires the graphicx package
\end{figure}


\begin{equation*}
\alpha_k^c = \frac{1}{Z} \sum_i \sum_j \frac{\partial{y^c}}{\partial A^k_{i,j}}
\end{equation*}

\begin{equation*}
CAM=ReLu(\sum_k \alpha_k^c A^k)
\end{equation*}
}

\frame{
\frametitle{Visualization}
\framesubtitle{Network's attention : Class Activation Map}

The resulting heatmap gives us the focus of the network. \\~\\

And implicitly, a localization algorithm !


\begin{table}
\centering
\begin{tabular}{cc}
   \includegraphics[width=5cm]{front.png} % requires the graphicx package
&
   \includegraphics[width=5cm]{heatmap2.png} % requires the graphicx package
\\
\end{tabular}
\end{table}
}


\section{Histogram of Oriented Gradients}

\frame{
\frametitle{HOGs}
\framesubtitle{A side approach}

I explored another approached for car-damage detection: the histogram of oriented gradients. \\~\\

Idea : manually compute features based on gradients in the image :

\begin{itemize}
\item 9 bins histogram of gradients for each image patch 
\item Bins corresponds to unsigned angles 
\item Dimensionality Reduction : 64X64X3 becomes 7x7x36 (divided by 7)
\end{itemize}

Then, train a SVM based on those features. \\~\\

Results:
\begin{itemize}
\item $98\%$ test accuracy on car/no car detection
\item $79\%$ test accuracy on damaged/clean detection
\end{itemize}
}


\frame{
\frametitle{HOGs}
\framesubtitle{Localization}

A sliding window is then applied on different scales of the input picture :

\begin{figure}[htbp]
   \centering
   \includegraphics[width=11cm]{hogs.png} % requires the graphicx package
\end{figure}

}

\section{Conclusion and Vision}
\frame{
\frametitle{Conclusions}

This is only the beginning, many refinements can still be brought. \\~\\

\begin{itemize}
\item Getting clean data is not easy !!!
\item Overfitting concerns in ConvNets (better regularization needed)
\item Better Generalization through more general dataset
\item Other Architectures could be investigated
\item Other technologies could be studied (e.g. per-pixel Segmentation)
\item Still working on getting the code tidy
\end{itemize}
}

\frame{
\frametitle{Vision}

Images are a important source of unused information \\~\\

Some (fancy) possible applications :

\begin{itemize}
\item Enhanced pricing evaluation
\item More Efficient handling of claims
\item Fraud Detection by combining picture angle and damage location
\item Fraud Detection by comparison with the official report
\item Assisted trial decisions (He was definitely drunk)
\item Many more !
\end{itemize}
}

\frame{
\frametitle{Hackathon}

Computer vision on damaged cars could be an interesting topic for an hackathon \\~\\

Recommendations :

\begin{itemize}
\item Provide participants with a clean dataset
\item Pictures should come with extra data such as model, severity,...
\item Many different approaches with same data : Regression, localization, detection, ... Stimulate creativity
\item A crash course on deep CNN and related Python libs could be a good idea for beginners
\item Training those models may take time, provide some computational power (Google Cloud ?)

\end{itemize}

}

\frame{
\frametitle{Thank You !}

\textbf{Thank you} all for this internship at AXA DataLab !

\vfill

\begin{flushright}
Any Questions ?
\end{flushright}

}



\end{document}
